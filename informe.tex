\documentclass{article}
\usepackage[utf8]{inputenc}

\title{Informe de la Tarea Investigativa II, presentado por los
estudiantes del equipo No. 30}
\author{Francisco Vicente Suárez Bellón C-212 \\
  Max Bengochea C-211}
\date{December 2022}

\begin{document}

\maketitle
\section*{Título: PERIODIC SOLUTIONS FOR SMALL AND LARGE DELAYS IN
A TUMOR-IMMUNE SYSTEM MODEL}
\subsection*{\centering Autores del articulo:\\
                 RADOUANE YAFIA
             }
\subsection*{\centering Revista: \\ 
                Electronic Journal of Differential Equations
             }
\subsection*{\centering Año: \\ 
                2006
             }
\subsection*{\centering Impacto: 1.282 }

\section{Introduction}
\noindent En este articulo se hace referencia a un modelo de tumor inmune que expresa su comportamiento mediante 
 ecuaciones diferenciales en este se centran principalmente en su análisis de estabilidad y su estudio de
    soluciones periódicas. Para ello se hace uso de la teoría de la estabilidad de Lyapunov. El sistema el cual analizan en 
    profundidad es un sistema de ecuaciones diferenciales no ordinarias las cuales se 
    salen del objeto de este trabajo las cuales son:
    \centering
    \subsection*{Ecuaciones no ordinarias}
        $\frac{d x}{d t} $ =$\sigma + \omega x(t- \tau )-
        \delta x
        $
        //
        $\frac{d y}{d t}$=$\alpha y(1-\beta y)-xy$
    


\end{document}
