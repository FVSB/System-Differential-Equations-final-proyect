\documentclass{article}
\usepackage[utf8]{inputenc}
\usepackage{array}
\usepackage{graphicx}
\title{Informe de la Tarea Investigativa II, presentado por los
estudiantes del equipo No. 30}
\author{Francisco Vicente Suárez Bellón C-212 \\
  Max Bengochea C-211}
\date{December 2022}

\begin{document}

\maketitle
\section*{Título: PERIODIC SOLUTIONS FOR SMALL AND LARGE DELAYS IN
A TUMOR-IMMUNE SYSTEM MODEL}
\subsection*{\centering Autores del articulo:\\
                 RADOUANE YAFIA
             }
\subsection*{\centering Revista: \\ 
                Electronic Journal of Differential Equations
             }
\subsection*{\centering Año: \\ 
                2006
             }
\subsection*{\centering Impacto: 1.282 }

\section{Introduction}
\noindent En este articulo se hace referencia a un modelo de tumor inmune que expresa su comportamiento mediante 
 ecuaciones diferenciales en este se centran principalmente en su análisis de estabilidad y su estudio de
    soluciones periódicas. Para ello se hace uso de la teoría de la estabilidad de Lyapunov. El sistema el cual analizan en 
    profundidad es un sistema de ecuaciones diferenciales no ordinarias las cuales se 
    salen del objeto de este trabajo las cuales son:
    \centering
    \subsection*{Ecuaciones no ordinarias}
      $\frac{d x}{d t} $ =$\sigma + \omega x(t- \tau )y(t- \tau )-
        \delta x
        $
        \\
        $\frac{d y}{d t}$=$\alpha y(1-\beta y)-xy$



Sin embargo tambien nos proporcionan otro sistema de ecuaciones diferenciales ordinarias las cuales son el objeto de estudio 
en este trabajo 
\centering
   \subsection*{Sistema a trabajar}
        $\frac{d x}{d t} $ =$\sigma + \omega xy-
            \delta x
            $
            \\
            $\frac{d y}{d t}$=$\alpha y(1-\beta y)-xy$

\section{Metodología}
\noindent 
  Para poder analizar este sistema no lineal de ecuaciones diferenciales ordinarias usaremos los recursos numericos como Runge-
  Kutta 4 y el método de Euler Modificado además de utilizar recursos computacionales para ilustar los resultados obtenidos mediante
  gráficas.
  
  Comenzamos buscando los posibles puntos de equilibrio de este sistema de ecuaciones diferenciales ordinarias
  para ello buscamos para que valores ambas ecuaciones se hacen cero simultáneamente luego para obtener su análisis de estabilidad
  usaremos el teorema de Hartman–Grobman\\
  Para aplicar las hipótesis del teorema antes descrito es necesario "linealizar" el sistema dado que este no 
  es lineal por lo que se hace uso de la siguiente matriz de jacobiano del sistema respecto a  $x$ y $y$ respectivamente
 \subsection*{Matriz de jacobiano}
    \[
    \left(
    \begin{array}{lc}
      y\omega-\delta & x\omega\\
      -y & -x-y\alpha\beta+\alpha(-y\beta+1)
    \end{array}
    \right)
    \]
  
  \centering  
  
  Nos centraremos en el análisis en el punto de equilibrio $(0,0)$ dado que en dicho punto la matriz se convierte en una
  matriz diagonal lo cual facilita mucho el análisis de estabilidad de este punto de equilibrio.
  \subsection*{Matriz jacobina en (0,0)}
    \[
    \left(
    \begin{array}{lc}
      -\delta & 0\\
      0 & \alpha
    \end{array}
    \right)
    \]

  \noindent
    \subsection*{Análisis de estabilidad: \\ TEOREMA 2 Estabilidad de sistemas casi lineales}
      
      Sean $\lambda1$ y $\lambda2$ los eigenvalores de la matriz de coefiecientes del sistema lineal dado 
      asociado con el sistema casi lineal. Entonces:\\
      1. Si  $\lambda1$ $=$ $\lambda2$ son eigenvalores reales e iguales, por consiguiente el punto crítico 
      (0, 0)  es un nodo o un punto espiral, y es asintóticamente estable si $\lambda1$
      $=$  $\lambda2$ $<$ 0, e inestable si $\lambda1$ $=$  $\lambda2$ $>$ 0.\\
      2. Si $\lambda1$ y $\lambda2$  son imaginarios puros, entonces (0, 0) es un centro o un punto espiral, 
      y puede ser asintóticamente estable, estable o inestable.\\

      3. En caso contrario —esto es, que  $\lambda1$ y  $\lambda2$ no sean reales e iguales o imaginarios 
      puros—, el punto crítico (0, 0) del sistema casi lineal  es del mismo 
      tipo y con la misma estabilidad que el punto crítico (0, 0) del sistema lineal 
      asociado

      %punto a=-1 d=1
    \section{Resultados}
       El tipo de estabilidad del punto (0,0) dependerá de los valores de: $\delta$ y $\alpha$
       \subsection{$\delta$=1 y $\alpha$=-1} 
         La matriz del jacobiano quedaria
         \[
          \left(
          \begin{array}{lc}
            -1 & 0\\
            0 & -1
          \end{array}
          \right)
          \]         
          Con lo que es un nodo o punto espiral y asintóticamente estable 
          \subsection*{Planos Fase del punto }
          \noindent
          \includegraphics{Isoclinas de a=-1 y d=1.jpg}
           \includegraphics{Campo Vectoraal para a=-1 y d=1.jpg}\\
           
           %Punto a=1 d=-1
      \subsection*{$\delta$=-1 y $\alpha$=1}
        El tipo de estabilidad del punto (0,0) dependerá de los valores de: $\delta$ y $\alpha$
        La matriz del jacobiano quedaría
        \[
         \left(
         \begin{array}{lc}
           1 & 0\\
           0 & 1
         \end{array}
         \right)
         \]         
         Con lo que es un nodo o punto espiral y asintóticamente inestable
         \subsection*{Planos Fase del punto }
         \noindent
         \includegraphics{isoclinas 2jpg.jpg}
          \includegraphics{Campo vectorail de d=-1 a1 .jpg}
        
          %Punto a=-2 d=1
      \subsection*{$\delta$=1 y $\alpha$=-2}
       
        La matriz del jacobiano quedaría
        \[
         \left(
         \begin{array}{lc}
           -1 & 0\\
           0 & -2
         \end{array}
         \right)
         \]         
         Con lo que es un nodo impropio estable
         \subsection*{Planos Fase del punto }
         \noindent
         \includegraphics{isoclinas d=1 a=-2.jpg}
          \includegraphics{Diagrama de fases para a-2 d1.jpg}


           %Punto a=1 d=1
     \subsection*{$\delta$=1 y $\alpha$=1}
       
        La matriz del jacobiano quedaría
        \[
         \left(
         \begin{array}{lc}
           -1 & 0\\
           0 & 1
         \end{array}
         \right)
         \]         
         Punto silla inestable
         \subsection*{Planos Fase del punto }
         \noindent
         \includegraphics{punto silla inestable isoclinas.jpg}
          \includegraphics{punto silla inestable.jpg}
           
         %punto a=1 y d=-2
   \subsection*{$\delta$=1 y $\alpha$=1}
       
        La matriz del jacobiano quedaría
        \[
         \left(
         \begin{array}{lc}
           2 & 0\\
           0 & 1
         \end{array}
         \right)
         \]         
         Nodo impropio inestable
         \subsection*{Planos Fase del punto }
         \noindent
         \includegraphics{isoclinas d=-2 a=1.jpg}
          \includegraphics{vecotrial d=-2 a=1.jpg}
           
\end{document}

  
